% Are Oelsner
% Simulations Final Lab
\documentclass[11pt,addpoints,answers]{exam}
\input{hwheader}

\hwheader{Final Lab}
{Are Oelsner}
{Spring 2018, Lawson}
{CS 326, Simulations}
{Due: 12:00pm Wednesday, May 2}

\begin{document}

\noindent
\paragraph{Overview:} You will incorporate input modeling techniques to use the
Tyler's Grill data in a multiple-server queueing model, and then use output
analysis, including batch means, to recommend a number of servers for Tyler's
Grill.

\vspace*{6pt}
\centerline{\textbf{This work must be done on your own, not in a group.}}

\begin{questions}
  \question Read Section 5.2.2 of the textbook. 
  Then inspect the source code for the existing {\tt msq} function from the
  {\tt simEd} library, which implements a multiple-server queue model.

  Benchmark the {\tt msq} implementation.  
  Do so by providing evidence that output from {\tt msq} 
  using one server matches output from the {\tt ssq} implementation.
  \begin{solution}
    I ran both msq and ssq with the same seed and a maxArrivals of 100 and
    these are the first six sojourn times outputted for each:
    \begin{itemize}
      \item {\tt msq}: 0.01783502 1.64360925 1.38217328 2.46898049 2.07581526
        1.71455047
      \item {\tt ssq}: 0.01783502 1.64360925 1.38217328 2.46898049 2.07581526
        1.71455047
    \end{itemize}
  \end{solution}
     
  \question Use the method of moments (MOM) to separately fit a {\it gamma}\,($k$,$\Theta$)
  and a {\it lognormal}\,($\mu$,$\sigma^2$) distribution to the Tyler's service data.
  \begin{enumerate}
      
    \item In class, we worked through the details of MOM for {\it gamma}.
      In your write-up, provide the mathematical details and estimates
      for $k$ and $\Theta$ determined using {\tt tylersGrill\$serviceTimes}.
      \begin{solution}
      \end{solution}
      
    \item To use MOM for determining parameter estimates for the {\it lognormal}
      distribution, the easiest approach is to first take the log of the
      service data and then fit the {\it normal} distribution using MOM.
      (This is equivalent to fitting the {\it lognormal} distribution.)
      In your write-up, provide the mathematical details, and estimates
      for $\mu$ and $\sigma$ determined using {\tt tylersGrill\$serviceTimes}.
      \begin{solution}
      \end{solution}
      
    \item Provide an appropriately labeled graphic that includes a plot of
      the ecdf of the service data (recall use of {\tt plot.stepfun}), and cdf
      curves of the fitted {\it gamma} and {\it lognormal} distributions
      superimposed.
      Use solid blue and line width 2 for {\it gamma}; use dotted red and line
      width 2 for {\it lognormal}.
      \begin{solution}
        %\includegraphics[scale=0.5]{code/arrival_process.pdf}
      \end{solution}
      
    \item Use the Kolmogorov-Smirnov goodness of fit test to evaluate the
      fits of the fitted {\it gamma} and {\it lognormal} distributions.
      Using this test, which is determined to be the better fit?
      Report the $D$ statistic for each.
      \begin{solution}
      \end{solution}
      
    \item Modify the service method in your {\tt msq} code to use service
      times drawn from the best fitting distribution above.
  \end{enumerate}
   
  \question Extend Algorithm 9.3.3 from the textbook to work in a next-event
  fashion.  That is, Algorithm 9.3.3 as given produces a list of arrival times.
  You must implement a method that, on each call, will remember the last
  arrival and produce the next interarrival using the cumulative event rate
  function produced by the Tyler's Grill arrival data.
  \begin{enumerate}
    \item Include this code as a function that can be passed as the
      interarrival function to {\tt msq}.
    \item Provide convincing graphical evidence that your arrival-process
      function, in isolation, produces output consistent with the collected
      Tyler's Grill arrival data.
      An example is provided below, showing five different single-day arrival
      process sequences superimposed on the original data.
      \begin{solution}
        \begin{center}
          %\includegraphics[scale=0.5]{code/arrival_process.pdf}
        \end{center}
      \end{solution}
  \end{enumerate}
  
  \question Experiment with different numbers of servers in the context of the
  arrival and service processes above.  Provide meaningful statistics, with
  confidence intervals.  In particular, use the method of batch means to
  produce 95\% confidence intervals for the average sojourn time of customers
  in the system.  (Since you do not know the true distribution of sojourn times
  here, to convince yourself of correctness you should compute several such
  intervals and look for overlap among the intervals.  Note that, in this
  context, ``several'' need not be 100, as required in the batch means lab.)
  \begin{solution}
  \end{solution}
  
  \question Provide discussion that could be used to convince Tyler's Grill
  management that a particular number of servers, fixed throughout the day,
  could be used to provide ``reasonable'' service based on the collected
  arrival and service data.
  Remember that any recommendation will involve tradeoffs --- additional
  servers do not come free of charge.
  \begin{solution}
  \end{solution}
\end{questions}

 \paragraph{Submitting:}
 Include your source code and write-up.
 Your write-up must include:
    \vspace*{-10pt}
    \begin{itemize}
      \setlength{\itemsep}{0pt}
      \item discussion of the next-event msq implementation (event types,
        algorithms for those event types, etc.), and
        evidence benchmarking your {\tt msq} against {\tt ssq};
      \item appropriate derivations, graphics, and discussion used in fitting
        the service data;  
      \item appropriate discussion and graphics associated with the
        non-parametric approach for generating arrival data;
      \item appropriate numerical and/or graphical evidence of your
        experimentation of different numbers of servers;
      \item convincing discussion of the recommended number of servers to use
        based on your simulation experimentation.
    \end{itemize}
    \vspace*{-10pt}
 Package your work into a tarball named similar to {\tt cmsc326\_final\_bo4pz.tgz} and
 drop into the {\tt final} folder within your shared Box folder for this course.


\end{document}

